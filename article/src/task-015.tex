\section{Постановка задачи}


\subsection{Условие}
Остановимся подробнее на задаче, поставленной перед нами. Формально, это проблема о классификации текстовых вопросов на два класса: \textbf{разрешенный контент}, \textbf{токсичный контент}. Существует довольно полное определение токсичности, которое Quora приводит в условии:

\begin{itemize}
\item
Не нейтральный тон
\subitem
    -- Намеренное преувеличение направленное на подчеркивание определенной точки зрения
\subitem
    -- Риторический вопрос, не предполагающий конкретного ответа
\item Пренебрежение и/или оскорбление
\subitem
    -- Дискриминационные высказывания, поощрение стереотипов о незащищенных группах
\subitem
    -- Пренебрежение и/или оскорбление в адрес конкретного человека или группы людей
\subitem
    -- Клевета или основанная на неподтвержденных фактах информация в отношении конкретного человека или группы людей
\item Очевидно ложные высказывания
\item Запрещенные темы
\subitem
    -- Темы 18+
\subitem
    -- Политика
\subitem
    -- Жестокое обращение с животными

\end{itemize}

\pagebreak

\subsection{Входные данные}

Организаторы данного kaggle-соревнования предлагают следующий набор датасетов:
\begin{itemize}
\item \textbf{train.csv} - данные для обучения, для которых известен результат идеальной классифицирующей модели. Представляет из себя csv-файл, в первой колонке которого содержится uuid вопроса, во второй - текстовая формулировка, в третьей - класс, к которому пренадлежит вопрос.

\item \textbf{test.csv} - данные, для которых необходимо предсказать классификацию. Представляет из себя csv-файл, в первой колонке которого содержится uuid вопроса, во второй - текстовая формулировка.\\

\item \textbf{GoogleNews-vectors-negative300.bin} - словарь векторизации слов, построенный на корпусе текстов GoogleNews. Размерность векторов - 300.
\item \textbf{glove.840B.300d.txt} - словарь векторизации слов, построенный на корпусе текстов Glove [4]. Размерность векторов - 300.
\item \textbf{paragram\_300\_sll999.txt} - словарь векторизации слов, построенный на корпусе текстов Paragram. Размерность векторов - 300.
\item \textbf{wiki-news-300d-1M.vec} - словарь векторизации слов, построенный на корпусе текстов Wiki News. Размерность векторов - 300.
\end{itemize}

По предоставленному набору данных можно понять, что организаторы предлагают в решении подумать в сторону векторизации слов, используя эти словари. Хорошая идея, вернемся к ней позже.


\pagebreak





