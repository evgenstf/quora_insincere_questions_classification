
\section{Заключение}

В качестве итога проделанной работы предлагаю еще раз кратко закрепить то, к чему мы пришли в ходе нее.

Была рассмотрена задача об определении токсичного контента на портале знаний \textbf{Quora}. 

Мы осознали необходимость разработки гибкой архитектуры для более эффективного развития проекта. Такой подход позволил более четко обозначить части решения и написать их независимо друг от друга. 

После того, как наметилась структура проекта, была реализованна математическая модель, которая решает поставленную задачу. 

На выходе мы получили готовый сервис, который ввиду своей модульности легко можно разделить на микросервисы и внедрить в промышленный процесс компании.
